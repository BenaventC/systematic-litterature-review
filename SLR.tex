% Options for packages loaded elsewhere
\PassOptionsToPackage{unicode}{hyperref}
\PassOptionsToPackage{hyphens}{url}
\PassOptionsToPackage{dvipsnames,svgnames,x11names}{xcolor}
%
\documentclass[
  9pt,
  ignorenonframetext,
  aspectratio=54,
]{beamer}
\usepackage{pgfpages}
\setbeamertemplate{caption}[numbered]
\setbeamertemplate{caption label separator}{: }
\setbeamercolor{caption name}{fg=normal text.fg}
\beamertemplatenavigationsymbolsempty
% Prevent slide breaks in the middle of a paragraph
\widowpenalties 1 10000
\raggedbottom
\setbeamertemplate{part page}{
  \centering
  \begin{beamercolorbox}[sep=16pt,center]{part title}
    \usebeamerfont{part title}\insertpart\par
  \end{beamercolorbox}
}
\setbeamertemplate{section page}{
  \centering
  \begin{beamercolorbox}[sep=12pt,center]{part title}
    \usebeamerfont{section title}\insertsection\par
  \end{beamercolorbox}
}
\setbeamertemplate{subsection page}{
  \centering
  \begin{beamercolorbox}[sep=8pt,center]{part title}
    \usebeamerfont{subsection title}\insertsubsection\par
  \end{beamercolorbox}
}
\AtBeginPart{
  \frame{\partpage}
}
\AtBeginSection{
  \ifbibliography
  \else
    \frame{\sectionpage}
  \fi
}
\AtBeginSubsection{
  \frame{\subsectionpage}
}

\usepackage{amsmath,amssymb}
\usepackage{lmodern}
\usepackage{iftex}
\ifPDFTeX
  \usepackage[T1]{fontenc}
  \usepackage[utf8]{inputenc}
  \usepackage{textcomp} % provide euro and other symbols
\else % if luatex or xetex
  \usepackage{unicode-math}
  \defaultfontfeatures{Scale=MatchLowercase}
  \defaultfontfeatures[\rmfamily]{Ligatures=TeX,Scale=1}
\fi
\usetheme[]{Boadilla}
\usefonttheme{serif}
% Use upquote if available, for straight quotes in verbatim environments
\IfFileExists{upquote.sty}{\usepackage{upquote}}{}
\IfFileExists{microtype.sty}{% use microtype if available
  \usepackage[]{microtype}
  \UseMicrotypeSet[protrusion]{basicmath} % disable protrusion for tt fonts
}{}
\makeatletter
\@ifundefined{KOMAClassName}{% if non-KOMA class
  \IfFileExists{parskip.sty}{%
    \usepackage{parskip}
  }{% else
    \setlength{\parindent}{0pt}
    \setlength{\parskip}{6pt plus 2pt minus 1pt}}
}{% if KOMA class
  \KOMAoptions{parskip=half}}
\makeatother
\usepackage{xcolor}
\newif\ifbibliography
\setlength{\emergencystretch}{3em} % prevent overfull lines
\setcounter{secnumdepth}{-\maxdimen} % remove section numbering


\providecommand{\tightlist}{%
  \setlength{\itemsep}{0pt}\setlength{\parskip}{0pt}}\usepackage{longtable,booktabs,array}
\usepackage{calc} % for calculating minipage widths
\usepackage{caption}
% Make caption package work with longtable
\makeatletter
\def\fnum@table{\tablename~\thetable}
\makeatother
\usepackage{graphicx}
\makeatletter
\def\maxwidth{\ifdim\Gin@nat@width>\linewidth\linewidth\else\Gin@nat@width\fi}
\def\maxheight{\ifdim\Gin@nat@height>\textheight\textheight\else\Gin@nat@height\fi}
\makeatother
% Scale images if necessary, so that they will not overflow the page
% margins by default, and it is still possible to overwrite the defaults
% using explicit options in \includegraphics[width, height, ...]{}
\setkeys{Gin}{width=\maxwidth,height=\maxheight,keepaspectratio}
% Set default figure placement to htbp
\makeatletter
\def\fps@figure{htbp}
\makeatother

\makeatletter
\makeatother
\makeatletter
\makeatother
\makeatletter
\@ifpackageloaded{caption}{}{\usepackage{caption}}
\AtBeginDocument{%
\ifdefined\contentsname
  \renewcommand*\contentsname{Table of contents}
\else
  \newcommand\contentsname{Table of contents}
\fi
\ifdefined\listfigurename
  \renewcommand*\listfigurename{List of Figures}
\else
  \newcommand\listfigurename{List of Figures}
\fi
\ifdefined\listtablename
  \renewcommand*\listtablename{List of Tables}
\else
  \newcommand\listtablename{List of Tables}
\fi
\ifdefined\figurename
  \renewcommand*\figurename{Figure}
\else
  \newcommand\figurename{Figure}
\fi
\ifdefined\tablename
  \renewcommand*\tablename{Table}
\else
  \newcommand\tablename{Table}
\fi
}
\@ifpackageloaded{float}{}{\usepackage{float}}
\floatstyle{ruled}
\@ifundefined{c@chapter}{\newfloat{codelisting}{h}{lop}}{\newfloat{codelisting}{h}{lop}[chapter]}
\floatname{codelisting}{Listing}
\newcommand*\listoflistings{\listof{codelisting}{List of Listings}}
\makeatother
\makeatletter
\@ifpackageloaded{caption}{}{\usepackage{caption}}
\@ifpackageloaded{subcaption}{}{\usepackage{subcaption}}
\makeatother
\makeatletter
\@ifpackageloaded{tcolorbox}{}{\usepackage[many]{tcolorbox}}
\makeatother
\makeatletter
\@ifundefined{shadecolor}{\definecolor{shadecolor}{rgb}{.97, .97, .97}}
\makeatother
\makeatletter
\makeatother
\ifLuaTeX
  \usepackage{selnolig}  % disable illegal ligatures
\fi
\IfFileExists{bookmark.sty}{\usepackage{bookmark}}{\usepackage{hyperref}}
\IfFileExists{xurl.sty}{\usepackage{xurl}}{} % add URL line breaks if available
\urlstyle{same} % disable monospaced font for URLs
\hypersetup{
  pdftitle={Systematic Litterature Review},
  pdfauthor={Christophe Benavent et Olivier Caron},
  colorlinks=true,
  linkcolor={Maroon},
  filecolor={Maroon},
  citecolor={Blue},
  urlcolor={Blue},
  pdfcreator={LaTeX via pandoc}}

\title{Systematic Litterature Review}
\author{Christophe Benavent et Olivier Caron}
\date{January 10, 2023}
\institute{Université Dauphine-PSL - DRM - Acss}

\begin{document}
\frame{\titlepage}
\ifdefined\Shaded\renewenvironment{Shaded}{\begin{tcolorbox}[boxrule=0pt, frame hidden, interior hidden, sharp corners, enhanced, breakable, borderline west={3pt}{0pt}{shadecolor}]}{\end{tcolorbox}}\fi

\renewcommand*\contentsname{Sommaire}
\begin{frame}[allowframebreaks]
  \frametitle{Sommaire}
  \tableofcontents[hideallsubsections]
\end{frame}
\hypertarget{introduction}{%
\section{Introduction}\label{introduction}}

\begin{frame}{Introduction}
Systematic reviews are different from traditional literature reviews
because they aim to identify all studies that address a specific
question.

\begin{itemize}
\item
  A quantitative dimension because large amoubt of references.
\item
  Text mining methods because we deal mainly with text.
\item
  Toward reproducible survey.
\end{itemize}

Three questions

\begin{itemize}
\item
  How to constitute the data set ?
\item
  How analyse Authors ?
\item
  How to Analyse contents ?
\end{itemize}
\end{frame}

\hypertarget{data-sets}{%
\section{Data sets}\label{data-sets}}

\begin{frame}{A selection Process}
\protect\hypertarget{a-selection-process}{}
\begin{figure}

{\centering \includegraphics{./images/fig-1-full.png}

}

\caption{Figure 1 : An SLR process source :
https://doi.org/10.7717/peerj-cs.695/fig-1}

\end{figure}
\end{frame}

\begin{frame}{Manual extraction}
\protect\hypertarget{manual-extraction}{}
\begin{itemize}
\item
  playing with zotero and ebsco
\item
  Look at scopus
\end{itemize}
\end{frame}

\begin{frame}{Using API}
\protect\hypertarget{using-api}{}
https://www.bibliometrix.org/home/

https://aurelien-goutsmedt.com/post/extracting-biblio-data-1/

https://github.com/sbegueria/bibliometRics

https://github.com/ropensci/rcrossref : le plus important ?
\end{frame}

\hypertarget{networks-analysis}{%
\section{Networks analysis}\label{networks-analysis}}

\begin{frame}{l'exemple d'un réseau d'auteurs}
\protect\hypertarget{lexemple-dun-ruxe9seau-dauteurs}{}
PMP
\end{frame}

\begin{frame}{L'exemple d'un réseau de citation}
\protect\hypertarget{lexemple-dun-ruxe9seau-de-citation}{}
Un petit exemple par olivier
\end{frame}

\hypertarget{an-embeddings-approach}{%
\section{An Embeddings Approach}\label{an-embeddings-approach}}

\begin{frame}{An Embeddings Approach}
On reprend le code DM
\end{frame}

\hypertarget{section}{%
\section{}\label{section}}

\begin{frame}{}
Conclusion

\begin{itemize}
\item
  Une autre organisation de la revue de littérature qui demande un
  effort collectif
\item
\end{itemize}
\end{frame}



\end{document}
